\documentclass[a4paper,11pt]{dinbrief} 
\usepackage[utf8]{inputenc}
\usepackage[ngerman]{babel}
\usepackage[autostyle=true,german=quotes]{csquotes}
\setlength{\headheight}{1.1\baselineskip}
\usepackage{scrdate}
\usepackage{footnote}
\usepackage{scrlayer-scrpage}
\setlength{\headsep}{10mm}
\pagestyle{scrheadings}
\pagenumbering{gobble}
\hyphenation{Office Pro-gram-mie-rer Bash-Kennt-nis-sen}
\begin{document}
\subject{ [% bezeichnung %] }
\backaddress{Andreas Grell,
  Kühnehöfe 25,
  D-22761 Hamburg
} 
\signature{Andreas Grell} 
\Datum{\todaysname, \today} 
\address{Andreas Grell \\
  Softwareentwickler, Hispanist, Germanist \\
  Kühnehöfe 25 \\
  D-22761 Hamburg} 
\begin{letter}
   {
    [% firma %] \\
    [% IF anrede == "Herr" %]
      z.Hd. [% anrede %]n [% ansprechpartner %] \\
    [% END %]
    [% IF anrede == "Frau" %]
      z.Hd. [% anrede %]  [% ansprechpartner %] \\
    [% END %]
    [% strasse %] \\ 
    [% ort %]
   }
    %% Dies ist übrigens ein Befehl: \begin{letter}{Adresse}
    [% IF anrede == 'Herr' %]
      \opening{ Sehr geehrter [% anrede %] [% ansprechpartner %],}
    [% ELSIF anrede == 'Frau' %]
      \opening{ Sehr geehrte [% anrede %] [% ansprechpartner %],}
    [% ELSE %]
      \opening{ Sehr geehrte Damen und Herren, }
    [% END %]
      durch [% quelle %] auf [% firma %] aufmerksam geworden, möchte ich mich mit dem
      vorliegenden Anschreiben als \enquote{[% bezeichnung %]} bei Ihnen bewerben.
      
      Um Ihnen nun vorab eine kleine Vorstellung über mein bisheriges programmiererisches Wirken --
      chronologische Angaben hierzu siehe Lebenslauf und Zeugnisse (beides beiliegend) -- zu geben, sei mein
      Werdegang hier kurz skizziert:

      Begonnen habe ich, nach einem als C/C++-Programmierer unter Linux abgelegten Zertifikat, bei HAVI Solutions
      GmbH \& Co. KG, einem Hamburger Spezialisten für Datenmigration, als C-, Perl- und Bash-Programmierer, bei
      dem ich mich sowohl mit der Transformation von SQL-Quelldatenstrukturen (MySQL) in ebensolche
      Zieldatenstrukturen als auch der Extraktion von Bilddaten aus obsoleten Datencontainerformaten
      auseinandersetzte. Um die besagten Daten aus den mit herkömmlicher Software nicht mehr zugänglichen Containern
      zu extrahieren, schrieb ich in C eigene Programme. Nach der Extraktion verarbeitete ich die Bilddaten mit
      eigenen Python-Programmen weiter.

      Nach einem kurzen Intermezzo mit Django, PostgreSQL und Python bei der Firma nachtblau tv, arbeitete ich bei
      dem Börsenportal ARIVA in Kiel an der PDF-Erstellung von Produktinformationsblättern mit Hilfe von
      Perl und dem Perl-Framework Mason. Hierbei kam mir meine Vorliebe für reguläre Ausdrücke zugute, mit deren
      Hilfe ich die Korrektheit der immer wieder notwendigen Änderungen an den als HTML- oder PDF-Dokument
      auszuliefernden Produktinformationen gewährleistete.
      
      Daran schloss sich eine Projektarbeit an einem webbasierten Office-System bei der Hannoveraner Firma
      Delticom an, in deren Rahmen ich eine Google-Suggest-artige Funktionalität auf der Grundlage von Perl und AJAX
      entwickelte.

      Zuletzt wirkte ich bei DSS in Stockelsdorf an der Funktionserweiterung der Aboverwaltung mit. Die
      verwendeten Programmiersprachen waren hier Perl und das zugegebenermaßen schon etwas angestaubte COBOL,
      das ich eigens für die von mir zu bewältigenden Aufgaben erlernte. Als Datenbanksystem wurde dort die
      Berkeley Database verwendet.

      Nach einem von Dezember 2019 bis Februar 2020 absolvierten Lehrgang verfüge ich nun auch über Kenntnisse
      in Java. Nachdem ich mir bereits während des Lehrgangs die Grundlagen in Swing, JDBC und Mavenim Selbststudium
      angeeignet habe, beschäftige ich mich momentan auch mit JPA und Hibernate. Meine Lieblingssprachen aber
      sind weiterhin C und Perl: C, weil es schnell ist und trotz umfangreicher moderner Bibliotheken (GTK+,
      Datenbankanbindung, Netzwerk, glibc etc.) den Quellcode in kleine Programme kompiliert, Perl weil sein
      Umgang mit regulären Ausdrücken und der Komfort hinsichtlich der Anbindung an eine Vielzahl von Technologien
      (Datenbanken jeglicher Art, Templating, Zugriff auf Web-Inhalte, Netzwerk) nicht nur prozedural, sondern
      auch funktional und objektorientiert (Moose) ermöglicht.

      Als Sprach- und Literaturwissenschaftler sowie altgedienter Schlussredakteur mit jahrzehntelanger
      Redaktionserfahrung bei diversen Zeitschriftenverlagen (DER SPIEGEL, TV Movie, TV Spielfilm, PRINZ etc.)
      verfüge ich zudem über eine ausgeprägte Kommunikationsstärke (sowohl auf Deutsch als auch auf Spanisch
      und Englisch) und die Fähigkeit, selbst komplexe Sachverhalte allgemeinverständlich sowohl schriftlich als
      auch mündlich auch einem technisch weniger versierten Publikum zu vermitteln. Als Text-Layout-System
      verwende ich mittlerweile bevorzugt \LaTeXe, weil es Texte akkurater formatiert als die gängigen
      Office-Anwendungen -- dieser mit Perl-Template-Technologie und pdflatex unter Linux (Raspbian) erzeugte
      Text mag als ein Beispiel hierfür dienen.

      Dass ich nicht nur bei der Programmierung, sondern auch beim Erstellen von Texten die Vorteile des
      hervorragenden Versionskontrollsystems Git nutze, gehört bei mir seit Jahren zum Alltag.
      
      Verfügbar bin ich ab sofort für ein Bruttojahresgehalt von [% gehalt %] Euro.

      Unter https://github.com/phoiniks können sie im Verlauf von Projekten entstandene Codeschnipsel besichtigen.
      Sollten Ihnen mein hier freilich nur grob umrissenes Profil und das erwähnte Softwarematerial zusagen, so würde
      ich mich freuen, Sie und Ihr Unternehmen bei einem unverbindlichen Gespräch kennenlernen zu dürfen und verbleibe
      bis auf Weiteres


\closing{mit freundlichen Grüßen}
\par
\encl{Lebenslauf und Zeugnisse} 
\end{letter} 
\end{document} 
